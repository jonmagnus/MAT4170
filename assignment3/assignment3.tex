\documentclass{article}
\usepackage{assignment}

\title{MAT4170 -- Assignment 3}
\author{Jon-Magnus Rosenblad}
\date{}

\begin{document}
\maketitle

\section*{Problem 1 (3.2)}
Let $\mathbf t = (0,0,0,1,3,4,5,5)$.
The existence of the four cubic splines $\{B_{i,3}\}_{i=1}^4$ is due to our regularization not exceeding the degree of our splines.
There are $8 - (3 + 1) = 4$ of them because we have $8$ knots.
Their linear independence on $[1,3]$ follow from lemma 3.8.

The splines $\hat B_{1,3}$ and $\hat B_{2,3}$ are naturally identified with $B_{2,3}$ and $B_{3,3}$ respectively,
so they must also be linearly independent as a subset of a linearly independent set.

\section*{Problem 2 (4.5)}
Lemma 4.2 states that if $\mathbf \tau$ is a knot vector with at least $d + 2$ knots,
and it's knots occur as a subsequence of another knot vector $\mathbf t$,
then $\mathbb S_{d,\mathbf\tau}\subset \mathbb S_{d,\mathbf t}$.

We have already proven the case when $\mathbf \tau$ is $d + 1$-regular, and by extension $\mathbf t$ must also be at least $d + 1$-regular,
and we will use this in our proof of the general case.

Let $\mathbf \tau^\prime$ and $\mathbf t^\prime$ denote the $d + 1$-regularizations of $\mathbf \tau$ and $\mathbf t$ respectively.
We have a natural embedding of $\mathbb S_{d, \mathbf \mu}\subset \mathbb S_{d, \mathbf \mu^\prime}$ for $\mathbf \mu=\mathbf \tau, \mathbf t$
given by padding with zeros at the ends of the coefficient vectors.
By our restricted variant of the lemma we may take any $f\in\mathbb S_{d, \mathbf \tau}$,
lift it to $\mathbb S_{d, \mathbf \tau^\prime}$ and extend it to $\mathbb S_{d, \mathbf t^\prime}$.
It remains to show that we may restrict it again to $\mathbb S_{d, \mathbf t}$.

By the construction of our extensions $\mathbb S_{d,\mathbf\mu^\prime}\supset \mathbb S_{d, \mathbf \mu}$,
$f$ would not obtain any nonzero coefficients on our regularization,
and as no knot is inserted between the knots added in the regularization in the embedding $\mathbb S_{d,\mathbf \tau^\prime}\subset \mathbb S_{d,\mathbf t^\prime}$,
we would not obtain in any nonzero coefficients outside the span of our original coefficients.
It follows that the projection of $f$ from $\mathbb S_{d,\mathbf t^\prime}\to \mathbb S_{d,\mathbf t}$ does not alter $f$.

\section*{Problem 3 (4.6)}



\end{document}
