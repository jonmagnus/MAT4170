\documentclass{article}

\usepackage{assignment}

\author{Jon-Magnus Rosenblad}
\title{MAT4170 \-- Assignment \#1}

\begin{document}
\maketitle

\section*{Problem 1.3.a}

We have a set of knots $\vec t = (t_0, t_1, t_2, t_3)$
and set of points $(\vec c_0, \vec c_1, \vec c_2, \vec c_3)$.
We define a recursive relation 
$\vec p_{j,d}(t) = (1 - t) \vec p_{j - 1, d - 1}(t)
+ t \vec p_{j, d - 1}(t)$.

From degree $0$ our computation becomes
\begin{equation*}
\begin{aligned}
    \vec p_{i,0} &= \vec c_i\\
    \vec p_{1,1} &= (1 - t)\vec c_0 + t\vec c_1\\
    \vec p_{2,1} &= (1 - t)\vec c_1 + t\vec c_2\\
    \vec p_{3,1} &= (1 - t)\vec c_2 + t\vec c_3\\
    \vec p_{2,2} &= (1 - t)\vec p_{1,1} + t\vec p_{2,1}\\
    &= (1 - t)^2\vec c_0 + 2(1 - t)t\vec c_1 + t^2\vec c_2\\
    \vec p_{3,2} &= (1 - t)\vec p_{2,1} + t\vec p_{3,1}\\
    &= (1 - t)^2\vec c_1 + 2(1 - t)t\vec c_2 + t^2\vec c_3\\
    \vec p_{3,3} &= (1 - t)\vec p_{2,2} + t\vec p_{3,2}\\
    &= (1 - t)^3\vec c_0 + 3(1 - t)^2t \vec c_1
    + (1 - t)t^2\vec c_2 + t^3\vec c_3.
\end{aligned}
\end{equation*}

We denote the coefficient of $c_i$ in $p_{3,3}$ by $\lambda_i$.
For $t\in[0,1]$ it is immediately clear that $\lambda_i(t)\geq 0$.
Furthermore we observe $\sum\lambda_i = ((1 - t) + t)^3 = 1$ for each $t$,
so we must have $\lambda_i(t)\leq 1$ for $t\in [0,1]$.
This prooves that $p_{3,3}(t)$ is a convex combination of $(\vec c_i)$.

\section*{Problem 2.1.h}

We have the following definition recursive relation
\begin{equation*}
    B[t_0,\ldots, t_{n + 1}](x)
    =\frac {x - t_0}{t_n - t_0}B[t_0,\ldots,t_n](x)
    +\frac {t_{n + 1} - x}{t_{n + 1} - t_1} B[t_1,\ldots,t_{n + 1}](x)
\end{equation*}

We get the following computation

\begin{equation*}
\begin{aligned}
    B[0,3,4,6](x)
    &= \frac 1 4 x B[0,3,4](x) - \frac 1 3 (x - 6) B[3,4,6](x)\\
    B[0,3,4](x)
    &= \frac 1 3 x B[0,3](x) - (x - 4) B[3,4](x)\\
    B[3,4,6](x)
    &= (x - 3) B[3,4](x) - \frac 1 2 (x - 6) B[4,6](x)
\end{aligned}
\end{equation*}

Substituting we get

\begin{equation*}
\begin{aligned}
    B[0,3,4,6](x)
    &= \frac 1 4 x B[0,3,4](x) - \frac 1 3 (x - 6) B[3,4,6](x)\\
    &=\begin{aligned}[t]
        &\frac 1 4 x \left(\frac 1 3 x B[0,3](x) - (x - 4) B[3,4](x)\right)\\
        &- \frac 1 3 (x - 6) \left((x - 3)B[3,4](x) - \frac 1 2 (x - 6) B[4,6](x)\right)
    \end{aligned}\\
    &=\begin{aligned}[t]
        &\frac 1 {12} x^2 B[0,3](x)\\
        &+ \frac 1 {12} \left(- 3x(x - 4) - 4(x - 6)(x - 3)\right)B[3,4](x)\\
        &+ \frac 1 6 (x - 6)^2 B[4,6](x)
    \end{aligned}\\
    &=\begin{aligned}[t]
        &\frac 1 {12} x^2 B[0,3](x)\\
        &+ \frac 1 {12} \left(-7x^2 + 48x - 72\right)B[3,4](x)\\
        &+ \frac 1 6 (x - 6)^2 B[4,6](x)
    \end{aligned}
\end{aligned}
\end{equation*}

\section*{Problem 2.5}

\begin{enumerate}
\item 
    Assume $B_{j,d}$ only depends on $t_j,\ldots, t_{j + d + 1}$,
    and $B_{j + 1,d}$ only depends on $t_{j + 1},\ldots, t_{j + d + 2}$.
    Sicne $B_{j,d + 1}$ only depends on $B_{j,d},B_{j + 1, d}, t_j, t_{j + 1}, t_{d + j + 1}$
    and $t_{d + j + 2}$,
    by induction it only depends on $t_j,\ldots, t_{j + d + 2}$.
    The base case is trivially satisfied.

\item
    The property of (a) follows immediately by induction as $B_{j,d}$
    is a linear combination of $B_{j,d - 1}$ and $B_{j + 1, d - 1}$.
    Property (b) follows from (a) and the fact that the sequence $(t_i)$ is increasing.

\item
    By the relation 
    $B_{j,d}(x)
    = \frac {x - t_j}{t_{j + d} - t_j} B_{j,d - 1}(x)
    + \frac {t_{j + d + 1} - x}{t_{j + d + 1} - t_{j + 1}} B_{j + 1, d - 1}(x)$,
    our inductive step boils down to prooving that the coefficients
    $\frac {x - t_j}{t_{j + d} - t_j}$ and
    $\frac {t_{j + d + 1} - x}{t_{j + d + 1} - t_{j + 1}}$
    are positive for $x\in(t_j, t_{j + d + 1})$.
    This follows from the fact that $(t_i)$ is increasing.

\end{enumerate}

\end{document}
